%%%%%%%%%%%%%%%%%%%%%%%%%%%%%%%%%%%%%%%%%
% Short Sectioned Assignment
% LaTeX Template
% Version 1.0 (5/5/12)
%
% This template has been downloaded from:
% http://www.LaTeXTemplates.com
%
% Original author:
% Frits Wenneker (http://www.howtotex.com)
%
% License:
% CC BY-NC-SA 3.0 (http://creativecommons.org/licenses/by-nc-sa/3.0/)
%
%%%%%%%%%%%%%%%%%%%%%%%%%%%%%%%%%%%%%%%%%

%----------------------------------------------------------------------------------------
%	PACKAGES AND OTHER DOCUMENT CONFIGURATIONS
%----------------------------------------------------------------------------------------

\documentclass[paper=a4, fontsize=11pt]{scrartcl} % A4 paper and 11pt font size

\usepackage[colorlinks=true, allcolors=red]{hyperref}
\usepackage[T1]{fontenc} % Use 8-bit encoding that has 256 glyphs
%\usepackage{fourier} % Use the Adobe Utopia font for the document - comment this line to return to the LaTeX default
\usepackage[english]{babel} % English language/hyphenation
\usepackage{amsmath,amsfonts,amsthm} % Math packages

\usepackage{sectsty} % Allows customizing section commands
% \allsectionsfont{\centering \normalfont\scshape} % Make all sections centered, the default font and small caps

\usepackage{tikz}
\usetikzlibrary{automata,positioning}

\usepackage{fancyhdr} % Custom headers and footers
\pagestyle{fancyplain} % Makes all pages in the document conform to the custom headers and footers
\fancyhead{} % No page header - if you want one, create it in the same way as the footers below
\fancyfoot[L]{} % Empty left footer
\fancyfoot[C]{} % Empty center footer
\fancyfoot[R]{\thepage} % Page numbering for right footer
\renewcommand{\headrulewidth}{0pt} % Remove header underlines
\renewcommand{\footrulewidth}{0pt} % Remove footer underlines
\setlength{\headheight}{13.6pt} % Customize the height of the header

\numberwithin{equation}{section} % Number equations within sections (i.e. 1.1, 1.2, 2.1, 2.2 instead of 1, 2, 3, 4)
\numberwithin{figure}{section} % Number figures within sections (i.e. 1.1, 1.2, 2.1, 2.2 instead of 1, 2, 3, 4)
\numberwithin{table}{section} % Number tables within sections (i.e. 1.1, 1.2, 2.1, 2.2 instead of 1, 2, 3, 4)

\setlength\parindent{0pt} % Removes all indentation from paragraphs - comment this line for an assignment with lots of text

\usepackage{mathtools}
\DeclarePairedDelimiter{\ceil}{\lceil}{\rceil}
\DeclarePairedDelimiter{\floor}{\lfloor}{\rfloor}

\newcommand{\heads}{\textsc{h}}
\newcommand{\tails}{\textsc{t}}

\newcommand{\logten}{\log}%\mathrm{log}\hspace{0.05in}}
\newcommand{\logtwo}{\lg}%\mathrm{lg}\hspace{0.05in}}
\newcommand{\loge}{\ln}%\mathrm{ln}\hspace{0.05in}}

\theoremstyle{definition}
\newtheorem*{solution}{Solution}

\usepackage{algpseudocode, algorithm}

%----------------------------------------------------------------------------------------
%	TITLE SECTION
%----------------------------------------------------------------------------------------

\newcommand{\horrule}[1]{\rule{\linewidth}{#1}} % Create horizontal rule command with 1 argument of height

\title{
\normalfont \normalsize
\textsc{ECS xxx \hfill Dept. of Computer Science, University of California, Davis} % Your university, school and/or department name(s)
\horrule{0.5pt} \\[0.4cm] % Thin top horizontal rule
\huge Homework \#0 \\ % The assignment title
\horrule{2pt} \\[0.5cm] % Thick bottom horizontal rule
}


\author{Your Name and Student ID} % Put your name here

\date{\today}

\begin{document}

\maketitle

Classmate collaborator(s): If you collaborated with fellow students, you \emph{must} list them here.

\section{First Problem}
    \begin{solution}
        Write your solution here.
        You insert math into a sentence using the dollar sign, like this: $f(x) = 2x$.
        Sometimes you need an entire line for an equation, so you use the equation environment, like this:
        \begin{equation*}
            2^{n+1} = \sum_{i=0}^{n} 2^{i}
        \end{equation*}
        If you wish to define an equation that spans multiple lines, use the align environment, like this:
        \begin{align*}
            % This percent denotes a comment.
            % Here, the & tells Latex where to align the text
            % The \\ is required for every line except the last, and it denotes a newline.
            1 + 2 + 2^2 + \ldots + 2^{k+1} &= (1 + 2 + 2^2 + \ldots + 2^k) + 2^{k+1}\\
                &= 2^{k+1} - 1 + 2^{k+1}\\
                &= 2 \cdot 2^{k+1} - 1\\
                &= 2^{k + 2} - 1
        \end{align*}
        If you wish to define a function with cases, use the cases environment, like this:
        \begin{equation*}
            f(n) =
                \begin{cases}
                    n   &   \text{if $n$ is even } \\
                    -n  &   \text{if $n$ is odd}
                \end{cases}
        \end{equation*}
    \end{solution}

\section{Second Problem}
    \begin{solution}
    Sometimes you will want to write proofs. Use the proof environment, like this:
    \begin{proof}
        Suppose that $f:A \to B$ and $g:B \to C$ are injective. Suppose that $x, y \in A$ and $x \neq y$. Then $f(x) \neq f(y)$ because $f$ is injective. Similarly, $g(f(x)) \neq g(f(y))$ because $g$ is injective. Therefore $g \circ f (x) \neq g \circ f (y)$ whenever $x \neq y$, so $g \circ f$ is injective.
    \end{proof}
    \end{solution}

    Sometimes you will want to write algorithms. Use the algorithmic environment, like this:
    \begin{algorithm}
        \caption{YourAlgorithm$(A, B)$}
        \begin{algorithmic}
          \For{$i = 1$ to $n$}
              \If{$A[i] > B[i]$}
                \State $A[i]$ is bigger!
              \Else
                \State $A[i]$ is not bigger!
              \EndIf
          \EndFor
        \State \Return \texttt{True}
        \end{algorithmic}
        \end{algorithm}

    Sometimes you will want to use tables. Use the tabular environment, like this:\\

      \begin{tabular}{|l|cr|}
      $a$ & $b$ & 10\\ \hline
      $e$ & 2 & $d$\\
      \end{tabular}\\

    The argument after the beginning of tabular specifies the columns. l for left justification, c for centered, r for right justification. The bar $|$ creates a vertical line in the table. $\backslash$hline creates a horizontal line in the table.

\section{Third Problem}
    Here is a brief, incomplete list of useful math symbols etc.
    \begin{enumerate}
        \item Exponential $\exp{n}$, $e^n$, $2^n$ etc.
        \item Logarithm $\log$
        \item Big-O $O(f(n))$, Big-Omega $\Omega(f(n))$, Big-Theta $\Theta(f(n))$
        \item Summation $\sum_{i=1}^n i$
        \item Function $f: A \to B$
        \item Union $A \cup B$, Intersection $A \cap B$, Complement $\overline{A}$, Set Difference $A \setminus B$
        \item Set membership $x \in A$
        \item Power Set $\mathcal{P}(A)$
    \end{enumerate}

\end{document}
